\chapter{Modal Particles and At-Issue Presuppositions}

\section{Reasoning}

\subsection{Presuppositions are non-at-issue, \textit{italic}}

Presupposition triggers require that their prejacent be entailed by the common ground \parencite{stalnaker2002common}. In other words,
utterances with presuppositions are only defined when the presupposition is already part of the
accepted facts. In (\ref{psp}), no truth-value can be assigned since, in the actual world, there is no king of
France, which causes a presupposition failure with respect to the definite determiner heading the
subject constituent. On the other hand, in the parallel case (\ref{psp-good}), the presupposition is
met and the utterance can be assigned a truth-value. \textsc{This text should be in small caps}.

\begin{exe}
    \ex \begin{xlist}
        \ex\label{psp}The king of France is bald.
        \ex\label{psp-good} The prince of Monaco is bald.
    \end{xlist}
\end{exe}

Taking this as a starting point, we can say that presuppositions (in the ideal case) refer to old
pieces of information. This being the case, a further property is revealed: presuppositions are
entailments that are not (usually) up for discussion; they are not at-issue
\parencite{aravind2017factivity}. To see this, consider the direct denial in (\ref{denial-psp})
where the only possible interpretation refers to the at-issue content of A's utterance -- the fact
that Peter gave up stripping -- but not the presupposed, non-at-issue component -- the fact Peter
used to strip \parencite[cf.][]{tonhauser2012diagnosing}.

\begin{exe}
    \ex\label{denial-psp}A:\@ Peter stopped stripping.

    B: \# That's not true.

    \textit{Intended}: Peter never stripped before.
\end{exe}

\begin{exe}
    \ex Lexikoneinträge

    \sem{not} = $\lambda p \in D_t$ . $p=0$

    \sem{Carla} = Carla

    \sem{invite} = $\lambda x \in D_e$ . [$\lambda y \in D_e$ . $y$ lädt $x$ ein]

    \sem{a} = $\lambda f \in D_{\langle e, t\rangle}$ . $[\lambda g \in D_{\langle e, t\rangle}$ . es gibt ein $x$, sodass $f(x)=1$ und $g(x)=1]$

    \sem{politician} = $\lambda x \in D_e$ . $x$ ist ein Politiker

    $\forall\! x [f(x) \rightarrow g(x)]$

    $\exists x [f(x) \land g(x)]$
\end{exe}

And the text below should be in typewriter font:

\begin{exe}
    \ex \texttt{\url{maik.thalmann@gmail.com}}
\end{exe}
